Een home automation-systeem dient om een woning te automatiseren. Dit kan variëren van heel eenvoudige taken, zoals een licht inschakelen wanneer een schakelaar wordt omgezet, tot complexere taken zoals het regelen van de temperatuur — en daarbij ook de ramen, verwarming en gordijnen — afhankelijk van allerlei factoren. Het opzetten van dergelijke automatisaties is echter niet zo vanzelfsprekend. De gebruiker moet enige kennis hebben van home automation en een beetje kunnen programmeren. De meeste mensen beschikken niet over deze kennis en kunnen daardoor niet ten volle profiteren van de voordelen die home automation biedt.

Wij ontwikkelden een gebruiksvriendelijke applicatie die het voor iedereen mogelijk moet maken om hun huis te automatiseren, zonder dat daarvoor diepgaande technische kennis vereist is. We deden dit door bestaande home automation-systemen onder de loep te nemen en te onderzoeken waar verbeteringen mogelijk waren. Daarbij maakten we gebruik van \fodot, een taal gebaseerd op eerste-orde logica. Deze taal kan worden gebruikt met de redeneermachine IDP-Z3, die redeneert over de in \fodot beschreven kennis om verschillende taken uit te voeren. De \fodot-taal is echter, net als andere talen in de wereld van home automation, niet eenvoudig te begrijpen of te gebruiken. Daarom onderzochten we mogelijke alternatieven voor \fodot, waaruit bleek dat een blocks-based editor een goede optie zou zijn.

Vervolgens ontwikkelden we een applicatie die gebruikmaakt van zelfgemaakte blokjes om automatisaties te creëren. De blokjes worden vertaald naar een \fodot-beschrijving, die vervolgens door IDP-Z3 gebruikt kan worden om de automatisaties uit te voeren. We besteedden daarbij bijzondere aandacht aan de gebruiksvriendelijkheid van de applicatie, omdat we willen dat iedereen ermee aan de slag kan.

Om dit te testen, voerden we een gebruikerstest uit met 14 proefpersonen. Hoewel de resultaten van deze test met enige voorzichtigheid ge\"interpreteerd moeten worden, wijzen ze erop dat de applicatie gebruiksvriendelijk is. De testpersonen gaven aan dat ze zich, met behulp van de applicatie, redelijk zelfzeker voelen in hun vermogen om hun huis te automatiseren.